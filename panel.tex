\documentclass[12pt,a4paper]{article}

% =====================================================================
%  PANEL FOTOGRAFICO
%  Proyecto: CREACION DE LOS SERVICIOS OPERATIVOS O MISIONALES
%  INSTITUCIONALES EN LA MUNICIPALIDAD DEL CENTRO POBLADO DE CCASA
% =====================================================================

\usepackage[utf8]{inputenc}
\usepackage[T1]{fontenc}
\usepackage[spanish,es-nodecimaldot]{babel}
\usepackage{graphicx}
\usepackage{geometry}
\usepackage{fancyhdr}
\usepackage[most]{tcolorbox}
\usepackage{xcolor}
\usepackage{microtype}
\usepackage{needspace}
\usepackage{ragged2e}
\usepackage{tabularx}
\usepackage{eso-pic}

% --- Margenes ---
\geometry{
    a4paper,
    left=2cm,
    right=2cm,
    top=4.2cm,
    bottom=2.5cm,
    headheight=3cm,
    headsep=0.5cm
}

% --- Ruta de imagenes ---
\graphicspath{{./}{IMAGENES/}}

% --- Encabezado ---
\pagestyle{fancy}
\fancyhf{}
\renewcommand{\headrulewidth}{0.5pt}
% Todo el encabezado en una sola celda centrada para evitar desborde
\fancyhead[C]{%
    \makebox[\textwidth][c]{%
        \includegraphics[width=2.5cm,height=2.1cm,keepaspectratio]{logo_muni.png}%
        \hfill
        \parbox[c]{8.8cm}{%
            \centering
            \fontsize{5.5}{6.8}\selectfont\textbf{%
            CREACI\'{O}N DE LOS SERVICIOS OPERATIVOS O MISIONALES
            INSTITUCIONALES EN LA MUNICIPALIDAD DEL CENTRO POBLADO
            DE CCASA DEL DISTRITO DE CHALLHUAHUACHO, PROVINCIA DE
            COTABAMBAS, DEPARTAMENTO DE APUR\'{I}MAC%
            }%
        }%
        \hfill
        \includegraphics[width=2.5cm,height=2.1cm,keepaspectratio]{logo_inviertepe.png}%
    }%
}
\fancyfoot[C]{\small\thepage}
\renewcommand{\footrulewidth}{0pt}

% =====================================================================
%  COMANDO PRINCIPAL: \panelfoto{archivo}{actividad}{lugar}{fecha}{comentario}
%  Estructura: tcolorbox exterior (imagen) con tcolorbox interior gris (datos)
% =====================================================================
\newcommand{\panelfoto}[5]{%
    \needspace{12cm}%
    \begin{tcolorbox}[
        colback=white,
        colframe=black!30,
        boxrule=0.5pt,
        arc=2mm,
        left=5pt, right=5pt, top=5pt, bottom=5pt
    ]
        \centering
        \includegraphics[width=\linewidth, height=12cm, keepaspectratio]{#1}\par
        \vspace{5pt}%
        % Cuadro interior gris para los datos de la imagen
        \begin{tcolorbox}[
            colback=gray!9,
            colframe=gray!40,
            boxrule=0.4pt,
            arc=1.5mm,
            left=7pt, right=7pt, top=6pt, bottom=6pt
        ]
            {\fontsize{9}{11}\selectfont
            \noindent\textbf{ACTIVIDAD:}~#2\par\vspace{2pt}
            \noindent\textbf{LUGAR:}~#3\par\vspace{2pt}
            \noindent\textbf{FECHA:}~#4\par\vspace{5pt}
            \noindent\textbf{Comentario:}~\RaggedRight #5\par}%
        \end{tcolorbox}%
    \end{tcolorbox}%
    \clearpage
}

% =====================================================================
\begin{document}

% --- Imagen de fondo inferior en todas las páginas ---
\AddToShipoutPictureBG{%
    \AtPageLowerLeft{%
        \includegraphics[width=\paperwidth]{inferior.png}%
    }%
}

% ===========================  PORTADA  ==============================
\begin{titlepage}
    \thispagestyle{fancy}
    \vspace*{3cm}
    \begin{center}
        {\fontsize{10}{12}\selectfont\textbf{PROYECTO DE INVERSI\'{O}N P\'{U}BLICA}\par}
        \vspace{0.6cm}
        \textcolor{black!30}{\rule{0.6\textwidth}{0.5pt}}\par
        \vspace{0.8cm}
        {\fontsize{22}{26}\selectfont\bfseries PANEL FOTOGR\'{A}FICO\par}
        \vspace{0.4cm}
        {\fontsize{12}{14}\selectfont Registro de Actividades de Formulaci\'{o}n\par}
        \vspace{1.2cm}
        \textcolor{black!30}{\rule{0.6\textwidth}{0.5pt}}\par
        \vspace{1.8cm}
        {\fontsize{9}{11}\selectfont\itshape
        ``CREACI\'{O}N DE LOS SERVICIOS OPERATIVOS O MISIONALES
        INSTITUCIONALES EN LA MUNICIPALIDAD DEL CENTRO POBLADO DE CCASA
        DEL DISTRITO DE CHALLHUAHUACHO, PROVINCIA DE COTABAMBAS,
        DEPARTAMENTO DE APUR\'{I}MAC''\par}
        \vspace{1.8cm}
        {\fontsize{10.5}{13}\selectfont
        \textbf{Comunidad Campesina de Ccasa}\par
        Distrito de Challhuahuacho\,---\,Provincia de Cotabambas\par
        Departamento de Apur\'{i}mac\par}
        \vspace{1cm}
        {\fontsize{10}{12}\selectfont Enero\,---\,Febrero~2026\par}
    \end{center}
\end{titlepage}

% =====================================================================
%                     REGISTRO FOTOGRAFICO
% =====================================================================

% FOTO 1 | 30/01/2026 - 10:44 hrs.
\panelfoto%
    {30_1_10_44_en_la_cancha_sintetica_de_la_comunidad_de_esta_al_no_contar_con_un_local_exclusivo_la_reunion_se_hizo_en_un_lugra_improvisado.jpg}%
    {Reuni\'{o}n General en Espacio Improvisado de la Comunidad Campesina de Ccasa}%
    {Cancha Sint\'{e}tica --- Comunidad Campesina de Ccasa, Challhuahuacho}%
    {30/01/2026 --- 10:44 hrs.}%
    {Ante la carencia de un local comunal exclusivo para reuniones, la Comunidad Campesina de
     Ccasa habilit\'{o} de manera provisional la cancha sint\'{e}tica como espacio de encuentro.
     En dicho lugar se desarroll\'{o} la reuni\'{o}n convocada en el marco del proceso de
     formulaci\'{o}n del proyecto, cont\'{a}ndose con la participaci\'{o}n activa de dirigentes
     y comuneros, quienes asumieron compromisos colectivos vinculados al avance de la iniciativa
     de inversi\'{o}n p\'{u}blica.}%

% FOTO 2 | 03/02/2026 - 11:29 hrs.
\panelfoto%
    {3_2_11_29_oficina_de_unidad_formuladora_cordinacion_con_el_presidente_de_la_comunidad_con_el_economista_alarcon_cabrera_alex.jpg}%
    {Coordinaci\'{o}n con el Presidente de la Comunidad --- Unidad Formuladora}%
    {Oficina de Unidad Formuladora --- Municipalidad Distrital de Challhuahuacho}%
    {03/02/2026 --- 11:29 hrs.}%
    {En las instalaciones de la Unidad Formuladora se llev\'{o} a cabo una reuni\'{o}n de
     coordinaci\'{o}n con el presidente de la Comunidad Campesina de Ccasa y el economista
     Alex Alarc\'{o}n Cabrera. Durante la sesi\'{o}n se revisaron los avances del proceso de
     formulaci\'{o}n del proyecto, se establecieron compromisos para la entrega oportuna de
     documentos requeridos por el equipo t\'{e}cnico y se verificaron los aspectos sociales
     necesarios para la viabilidad del expediente.}%

% FOTO 3 | 03/02/2026 - 11:29 hrs. (segunda toma)
\panelfoto%
    {3_2_11_29_oficina_de_unidad_formuladora_cordinacion_con_el_presidente_de_la_comunidad_de_ccasa_Wilfredo_asto_con_el_economista_alarcon_cabrera_alex.jpg}%
    {Coordinaci\'{o}n con el Presidente Wilfredo Asto --- Unidad Formuladora}%
    {Oficina de Unidad Formuladora --- Municipalidad Distrital de Challhuahuacho}%
    {03/02/2026 --- 11:29 hrs.}%
    {Segunda toma del encuentro sostenido entre el presidente de la Comunidad Campesina de Ccasa,
     se\~{n}or Wilfredo Asto, y el economista Alex Alarc\'{o}n Cabrera, en las instalaciones de
     la Unidad Formuladora. La sesi\'{o}n tuvo como prop\'{o}sito revisar aspectos t\'{e}cnicos
     y sociales del proyecto, as\'{i} como verificar el estado de los compromisos asumidos por
     la comunidad para la correcta formulaci\'{o}n del expediente de inversi\'{o}n p\'{u}blica.}%

% FOTO 4 | 09/02/2026 (registro general)
\panelfoto%
    {9_2_ENCUESTA.jpg}%
    {Aplicaci\'{o}n de Encuesta Socioecon\'{o}mica --- Comunidad Campesina de Ccasa}%
    {Comunidad Campesina de Ccasa --- Challhuahuacho}%
    {09/02/2026}%
    {Registro fotogr\'{a}fico correspondiente a las actividades de levantamiento de informaci\'{o}n
     socioecon\'{o}mica desarrolladas en la Comunidad Campesina de Ccasa. El proceso de encuestado
     permiti\'{o} recopilar datos esenciales sobre las condiciones de vida de la poblaci\'{o}n,
     incluyendo aspectos de vivienda, servicios b\'{a}sicos y din\'{a}mica econ\'{o}mica familiar,
     constituy\'{e}ndose en insumo fundamental para el diagn\'{o}stico del proyecto de inversi\'{o}n
     p\'{u}blica en formulaci\'{o}n.}%

% FOTO 5 | 09/02/2026 - 08:50 hrs.
\panelfoto%
    {9_2_8_50_KKenco_ccasa_equipo_de_trabajo_encargado_de_realizar_las_encuestas_organizandose_antes_de_separarse.jpg}%
    {Organizaci\'{o}n del Equipo de Trabajo --- Inicio de Jornada de Encuestas}%
    {Sector Kencco --- Comunidad Campesina de Ccasa}%
    {09/02/2026 --- 08:50 hrs.}%
    {El equipo t\'{e}cnico encargado del levantamiento de informaci\'{o}n socioecon\'{o}mica se
     organiz\'{o} en el sector Kencco de la Comunidad Campesina de Ccasa, previo al inicio de
     las actividades de campo. En esta instancia se distribuyeron las rutas de trabajo, se
     asignaron responsabilidades a cada auxiliar y se impartieron las \'{u}ltimas indicaciones
     metodol\'{o}gicas, garantizando la cobertura sistem\'{a}tica de todas las viviendas
     contempladas en el proceso de recojo de datos.}%

% FOTO 6 | 09/02/2026 - 09:04 hrs.
\panelfoto%
    {9_02_9_04_Comunidad_CCASA_primer_encuestado_economista_realizando-demostracin_a_los_auxiliares.jpg}%
    {Demostraci\'{o}n de Aplicaci\'{o}n de Encuesta --- Primer Encuestado}%
    {Comunidad Campesina de Ccasa --- Challhuahuacho}%
    {09/02/2026 --- 09:04 hrs.}%
    {El economista a cargo realiz\'{o} la primera encuesta socioecon\'{o}mica de la jornada,
     sirvi\'{e}ndola como caso demostrativo para los auxiliares del equipo de campo. Dicha acci\'{o}n
     permiti\'{o} estandarizar la metodolog\'{i}a de aplicaci\'{o}n del instrumento de recojo de
     informaci\'{o}n, asegurando la coherencia y calidad de los datos obtenidos durante toda la
     jornada de trabajo en la comunidad.}%

% FOTO 7 | 09/02/2026 - 09:29 hrs.
\panelfoto%
    {9_02_9_29_presidente_de_la_comunidad_de_ccasa_en_el_terreno_donde_se_se_ejecutara_el_proyecto_garantizando_la_deisponibilidad-del_terreno.jpg}%
    {Verificaci\'{o}n del Terreno Destinado al Proyecto --- Presidente de Ccasa}%
    {Terreno del Proyecto --- Comunidad Campesina de Ccasa}%
    {09/02/2026 --- 09:29 hrs.}%
    {El presidente de la Comunidad Campesina de Ccasa acompa\~{n}\'{o} al equipo t\'{e}cnico al
     terreno donde se ejecutar\'{a} el proyecto, reafirmando y garantizando formalmente la
     disponibilidad del \'{a}rea para la intervenci\'{o}n. Dicho acto constituye un elemento
     fundamental para la viabilidad social y territorial del proyecto, al demostrar el respaldo
     y compromiso de la m\'{a}xima autoridad comunal con el proceso de formulaci\'{o}n del
     expediente de inversi\'{o}n p\'{u}blica.}%

% FOTO 8 | 09/02/2026 - 09:59 hrs.
\panelfoto%
    {9_2_9_59_ENCUESTA_CASA_DE_adobe_servicio_de_agua_habitante_que_sabe_leer_idrigente_completando_encuesta.jpg}%
    {Aplicaci\'{o}n de Encuesta en Vivienda de Adobe con Acceso a Agua}%
    {Comunidad Campesina de Ccasa --- Challhuahuacho}%
    {09/02/2026 --- 09:59 hrs.}%
    {Se registra la aplicaci\'{o}n de la encuesta socioecon\'{o}mica en una vivienda de
     construcci\'{o}n tradicional de adobe con acceso a servicio de agua. El encuestado, dirigente
     comunal con capacidad lectora, complet\'{o} el cuestionario de manera aut\'{o}noma bajo la
     supervisi\'{o}n del equipo t\'{e}cnico, aportando informaci\'{o}n precisa sobre las condiciones
     habitacionales del hogar, datos de especial relevancia para el diagn\'{o}stico socioecon\'{o}mico
     del proyecto.}%

% FOTO 9 | 09/02/2026 - 10:20 hrs.
\panelfoto%
    {9_02_10_20_encuestas_en_el-sector_parte_alta-de_casa_donde-se_consentra_la_menro_cantidad_de-pobladores.jpg}%
    {Aplicaci\'{o}n de Encuestas en el Sector Parte Alta de Ccasa}%
    {Sector Parte Alta --- Comunidad Campesina de Ccasa}%
    {09/02/2026 --- 10:20 hrs.}%
    {El equipo t\'{e}cnico desarroll\'{o} el levantamiento de informaci\'{o}n en el sector parte alta
     de la Comunidad Campesina de Ccasa, zona que concentra la menor densidad poblacional del
     \'{a}mbito de intervenci\'{o}n. La aplicaci\'{o}n de encuestas en este sector permiti\'{o}
     obtener datos representativos de las condiciones socioecon\'{o}micas de los hogares ubicados
     en la periferia de la comunidad, asegurando la cobertura total requerida por la metodolog\'{i}a
     de formulaci\'{o}n del proyecto.}%

% FOTO 10 | 09/02/2026 - 10:24 hrs.
\panelfoto%
    {9_02_10_24_vievienda_de_adobe_sin_techo_condicion_de_abandono_pero_con_acceso-a_energia_electrica_vienvienda.jpg}%
    {Constataci\'{o}n de Vivienda de Adobe sin Techo en Condici\'{o}n de Abandono}%
    {Comunidad Campesina de Ccasa --- Challhuahuacho}%
    {09/02/2026 --- 10:24 hrs.}%
    {Durante el recorrido de campo se identific\'{o} una vivienda construida con adobe, sin techo
     y en aparente estado de abandono, aunque con conexi\'{o}n activa a la red de energ\'{i}a
     el\'{e}ctrica. Esta evidencia refleja la precariedad habitacional que afecta a parte de la
     poblaci\'{o}n y constituye informaci\'{o}n relevante para el diagn\'{o}stico sobre el
     d\'{e}ficit de infraestructura y la situaci\'{o}n de vulnerabilidad que sustenta la necesidad
     del proyecto de inversi\'{o}n p\'{u}blica.}%

% FOTO 11 | 09/02/2026 - 10:24 hrs. (segunda toma)
\panelfoto%
    {9_2_10_24_monunidad_ccasa_comunero_alfabetizado_contestando_las_preguntas_del_cuestionario.jpg}%
    {Aplicaci\'{o}n de Encuesta a Comunero Alfabetizado de Ccasa}%
    {Comunidad Campesina de Ccasa --- Challhuahuacho}%
    {09/02/2026 --- 10:24 hrs.}%
    {Un comunero con capacidad de lectura y escritura de la Comunidad Campesina de Ccasa respondi\'{o}
     de manera activa y detallada las preguntas del cuestionario socioecon\'{o}mico. El nivel de
     comprensi\'{o}n y participaci\'{o}n demostrado facilit\'{o} el recojo de informaci\'{o}n
     precisa y confiable, contribuyendo significativamente a la solidez del diagn\'{o}stico
     elaborado por el equipo t\'{e}cnico de formulaci\'{o}n.}%

% FOTO 12 | 09/02/2026 - 10:36 hrs.
\panelfoto%
    {9_02_10_36_vivienda_de_material-noble_con-CON-acceso_a_energia_electrica_agua_y_saneamiento_en-el_transcursode_la_elaboracion_de_la_encuesta.jpg}%
    {Constataci\'{o}n de Vivienda de Material Noble con Servicios B\'{a}sicos}%
    {Comunidad Campesina de Ccasa --- Challhuahuacho}%
    {09/02/2026 --- 10:36 hrs.}%
    {En el transcurso de la aplicaci\'{o}n de encuestas se verific\'{o} la existencia de una
     vivienda construida con material noble, dotada de acceso a energ\'{i}a el\'{e}ctrica, agua
     y saneamiento. Esta constataci\'{o}n evidencia la heterogeneidad de las condiciones
     habitacionales en la comunidad, mostrando el contraste entre viviendas consolidadas y las
     que presentan d\'{e}ficit de servicios b\'{a}sicos en el \'{a}mbito de intervenci\'{o}n del
     proyecto.}%

% FOTO 13 | 09/02/2026 - 10:39 hrs.
\panelfoto%
    {9_2_10_39_ENCUESTA.jpg}%
    {Levantamiento de Informaci\'{o}n Socioecon\'{o}mica --- Encuesta de Campo}%
    {Comunidad Campesina de Ccasa --- Challhuahuacho}%
    {09/02/2026 --- 10:39 hrs.}%
    {Registro fotogr\'{a}fico de la continuaci\'{o}n del proceso de levantamiento de informaci\'{o}n
     socioecon\'{o}mica en la Comunidad Campesina de Ccasa. El equipo t\'{e}cnico prosigui\'{o}
     con la aplicaci\'{o}n sistem\'{a}tica de los cuestionarios, recopilando datos sobre vivienda,
     servicios b\'{a}sicos, actividad econ\'{o}mica e ingresos familiares como insumos para el
     diagn\'{o}stico situacional que sustenta el proyecto de inversi\'{o}n p\'{u}blica.}%

% FOTO 14 | 09/02/2026 - 10:42 hrs.
\panelfoto%
    {9_02_10_42_CENTRO_DE_SALUD_de_la_comuniidada_campesina_de_ccasa_en_buen_estado_constrataciondelaexustencia_de_la_posta_medica_despues_de_las_encuestas.jpg}%
    {Constataci\'{o}n de la Infraestructura del Centro de Salud --- Posta M\'{e}dica de Ccasa}%
    {Centro de Salud (Posta M\'{e}dica) --- Comunidad Campesina de Ccasa}%
    {09/02/2026 --- 10:42 hrs.}%
    {Posterior al desarrollo de las encuestas, el equipo t\'{e}cnico const\'{a}t\'{o} la existencia
     del Centro de Salud de la Comunidad Campesina de Ccasa, el cual se encontraba en buen estado
     de conservaci\'{o}n. Este establecimiento constituye un equipamiento de salud relevante dentro
     del diagn\'{o}stico de servicios e infraestructura comunal, proporcionando informaci\'{o}n de
     contexto para la delimitaci\'{o}n del \'{a}mbito del proyecto de inversi\'{o}n.}%

% FOTO 15 | 09/02/2026 - 10:47 hrs.
\panelfoto%
    {9_02_10_47_kkenko_ccasa_la_visualizacion_de_una_vivienda_de_adobe_y_techo_de_calamina_ypuente__elaborado_sin_acceso_a_energia_electrica.jpg}%
    {Registro de Vivienda de Adobe con Techo de Calamina y Puente Artesanal sin Energ\'{i}a}%
    {Sector Kencco --- Comunidad Campesina de Ccasa}%
    {09/02/2026 --- 10:47 hrs.}%
    {En el sector Kencco se registr\'{o} la presencia de una vivienda construida con adobe y techo
     de calamina, acompa\~{n}ada de un puente artesanal elaborado por los propios moradores, en
     ausencia de acceso a la red de energ\'{i}a el\'{e}ctrica. Esta situaci\'{o}n ilustra el
     d\'{e}ficit de infraestructura b\'{a}sica existente en la comunidad y refuerza la justificaci\'{o}n
     t\'{e}cnica del proyecto de inversi\'{o}n p\'{u}blica.}%

% FOTO 16 | 09/02/2026 - 10:53 hrs.
\panelfoto%
    {9_02_10_53_econommista_alex_alarcon_cabrera_junto_a_la_entrada_del_coelgio_de_ccasa_comprobando_la_existencia_de_una.jpg}%
    {Verificaci\'{o}n de Infraestructura Educativa --- Colegio de la Comunidad de Ccasa}%
    {Instituci\'{o}n Educativa --- Comunidad Campesina de Ccasa}%
    {09/02/2026 --- 10:53 hrs.}%
    {El economista Alex Alarc\'{o}n Cabrera se desplaz\'{o} hasta la entrada del colegio de la
     Comunidad Campesina de Ccasa con el fin de verificar la existencia y el estado de la
     infraestructura educativa. Dicha verificaci\'{o}n forma parte del diagn\'{o}stico de
     equipamientos comunales que sustenta la formulaci\'{o}n del proyecto, registr\'{a}ndose que
     la instituci\'{o}n brinda servicios de educaci\'{o}n b\'{a}sica a la poblaci\'{o}n en edad
     escolar de la comunidad.}%

% FOTO 17 | 09/02/2026 - 10:53 hrs. (segunda toma)
\panelfoto%
    {9_02_10_53_segunda_toma_exclusiva_del-colegio_de_ccasa_que_funciona_como_inicial_primaria_y_secundaria.jpg}%
    {Registro Fotogr\'{a}fico de la Instituci\'{o}n Educativa Multigrado de Ccasa}%
    {Instituci\'{o}n Educativa --- Comunidad Campesina de Ccasa}%
    {09/02/2026 --- 10:53 hrs.}%
    {Segunda toma fotogr\'{a}fica de la Instituci\'{o}n Educativa de la Comunidad Campesina de
     Ccasa, la cual funciona como establecimiento multigrado atendiendo los niveles de educaci\'{o}n
     inicial, primaria y secundaria dentro de una misma infraestructura. Este registro documenta el
     estado del equipamiento educativo existente y su importancia como servicio esencial para la
     poblaci\'{o}n comunal en el \'{a}mbito del proyecto.}%

% FOTO 18 | 09/02/2026 - 11:08 hrs.
\panelfoto%
    {9_02_11_8_SSHH_del_local_improvisado_del_consejo_menor_SSHH_de_la_cancha_sintetica_infraestuctura_con_uso_exclusivo_de_actividades_deprtivas.jpg}%
    {Constataci\'{o}n de Servicios Higi\'{e}nicos --- Cancha Sint\'{e}tica del Consejo Menor}%
    {Cancha Sint\'{e}tica --- Local del Consejo Menor, Comunidad Campesina de Ccasa}%
    {09/02/2026 --- 11:08 hrs.}%
    {Se const\'{a}t\'{o} el estado de los servicios higi\'{e}nicos del local improvisado del Consejo
     Menor, ubicados en las instalaciones de la cancha sint\'{e}tica de la comunidad. Dicha
     infraestructura sanitaria tiene un uso exclusivo para actividades deportivas, evidenciando la
     inexistencia de instalaciones adecuadas para el funcionamiento administrativo y la atenci\'{o}n
     p\'{u}blica de la organizaci\'{o}n comunal, lo cual constituye uno de los elementos que motivan
     la necesidad del proyecto de inversi\'{o}n.}%

% FOTO 19 | 09/02/2026 (sin hora exacta)
\panelfoto%
    {9_02_consattacion_de_la_existencia_de_casa-_de_material_noble_en_el_tramnscurso_de_realizar_la_encuesta_material_noble_qone_wasi.jpg}%
    {Constataci\'{o}n de Vivienda de Material Noble --- Qo\~{n}e Wasi}%
    {Comunidad Campesina de Ccasa --- Challhuahuacho}%
    {09/02/2026}%
    {Durante el proceso de levantamiento de informaci\'{o}n se const\'{a}t\'{o} la existencia de
     viviendas construidas con material noble en la Comunidad Campesina de Ccasa, denominadas
     localmente ``Qo\~{n}e Wasi''. Este tipo de construcciones refleja el avance progresivo en
     las condiciones habitacionales de algunos comuneros y provee un indicador de contraste respecto
     a las edificaciones tradicionales de adobe predominantes en la zona, enriqueciendo el
     diagn\'{o}stico sobre la heterogeneidad socioecon\'{o}mica de la poblaci\'{o}n beneficiaria.}%

% FOTO 20 | 09/02/2026 - 11:26 hrs.
\panelfoto%
    {9_2_11_26_ENCUESTA.jpg}%
    {Aplicaci\'{o}n de Encuesta Socioecon\'{o}mica --- Continuaci\'{o}n de Jornada}%
    {Comunidad Campesina de Ccasa --- Challhuahuacho}%
    {09/02/2026 --- 11:26 hrs.}%
    {Registro correspondiente a la continuaci\'{o}n de las actividades de encuestado en la
     Comunidad Campesina de Ccasa durante las horas de la ma\~{n}ana. El equipo t\'{e}cnico
     mantuvo el ritmo de aplicaci\'{o}n sistem\'{a}tica de los cuestionarios, consolidando la
     base de datos socioecon\'{o}mica necesaria para el an\'{a}lisis de la demanda y la
     justificaci\'{o}n t\'{e}cnica del proyecto de inversi\'{o}n p\'{u}blica.}%

% FOTO 21 | 09/02/2026 - 11:27 hrs.
\panelfoto%
    {9_02_11_27_encuesta_a_comunero_dirigentel_sector_Kencco_kasa_opiniones_con_mas_relevancia_por_parete_de_uno_del_auxiliar_proxima_ing_Nayda_Quispe.jpg}%
    {Encuesta a Comunero Dirigente --- Sector Kencco de la Comunidad de Ccasa}%
    {Sector Kencco --- Comunidad Campesina de Ccasa}%
    {09/02/2026 --- 11:27 hrs.}%
    {La ingenieria auxiliar Nayda Quispe realiz\'{o} la encuesta a un comunero con rango dirigencial
     en el sector Kencco de la Comunidad Campesina de Ccasa. Las opiniones expresadas por el
     encuestado revistieron especial relevancia, al aportar una perspectiva de liderazgo comunitario
     que enriquece el diagn\'{o}stico participativo del proyecto. La visi\'{o}n del dirigente
     contribuy\'{o} a identificar las principales problem\'{a}ticas y demandas de servicios
     prioritarios de la zona de intervenci\'{o}n.}%

% FOTO 22 | 09/02/2026 - 15:16 hrs.
\panelfoto%
    {9_2_15_16_cordinacion_con_el_alcalde_del_consejp_menor_en_oficina_de_unidad-formuladora_economista_alex_y_el_ing_jhonfhay_kennedy.jpg}%
    {Coordinaci\'{o}n con el Alcalde del Consejo Menor --- Unidad Formuladora}%
    {Oficina de Unidad Formuladora --- Municipalidad Distrital de Challhuahuacho}%
    {09/02/2026 --- 15:16 hrs.}%
    {En horas de la tarde, el economista Alex y el ingeniero Jhonfhay Kennedy concurrieron a la
     Oficina de la Unidad Formuladora para sostener una reuni\'{o}n de coordinaci\'{o}n con el
     alcalde del Consejo Menor. En dicho encuentro se revisaron los avances del proyecto, se
     abordaron los compromisos institucionales pendientes y se acordaron las acciones necesarias
     para garantizar la continuidad del proceso de formulaci\'{o}n del expediente t\'{e}cnico
     de inversi\'{o}n p\'{u}blica.}%

% FOTO 23 | 19/02/2026 - 09:30 hrs.
\panelfoto%
    {19_02_9_30_reunicon_con_el_arquitecto_y_dirigentes_del_consejo_menor_con_economista_observand_en_pantalla_el_estado_del_proyecto.jpg}%
    {Reuni\'{o}n con el Arquitecto y Dirigentes del Consejo Menor --- Revisi\'{o}n del Proyecto}%
    {Oficina de Unidad Formuladora --- Municipalidad Distrital de Challhuahuacho}%
    {19/02/2026 --- 09:30 hrs.}%
    {Se desarroll\'{o} una reuni\'{o}n de trabajo con el arquitecto asignado al proyecto y los
     dirigentes del Consejo Menor, con la participaci\'{o}n del economista del equipo formulador.
     Durante la sesi\'{o}n se proyect\'{o} en pantalla el estado actualizado del expediente,
     permitiendo a los dirigentes conocer los avances t\'{e}cnicos, validar la informaci\'{o}n
     recogida y emitir observaciones pertinentes para la mejora del proyecto de inversi\'{o}n
     p\'{u}blica en su etapa de formulaci\'{o}n.}%

% FOTO 24 | 19/02/2026 - 09:37 hrs.
\panelfoto%
    {19_02_9_37_oficina_unidad_formuladora_cordianacion_con_el_ing_ambiental_Paul_VIdal_mamani_vera.jpg}%
    {Coordinaci\'{o}n con el Ing. Ambiental Paul Vidal Mamani Vera --- Unidad Formuladora}%
    {Oficina de Unidad Formuladora --- Municipalidad Distrital de Challhuahuacho}%
    {19/02/2026 --- 09:37 hrs.}%
    {En la Oficina de la Unidad Formuladora se llev\'{o} a cabo una sesi\'{o}n de coordinaci\'{o}n
     con el ingeniero ambiental Paul Vidal Mamani Vera. La reuni\'{o}n tuvo como objetivo revisar
     los aspectos ambientales del proyecto y alinear los criterios t\'{e}cnicos necesarios para el
     cumplimiento de los requisitos de evaluaci\'{o}n de impacto ambiental, componente indispensable
     en el proceso de formulaci\'{o}n y aprobaci\'{o}n del expediente en el marco del sistema
     Invierte.pe.}%

% FOTO 25 | Febrero 2026
\panelfoto%
    {oficina_de_unidad_formuladora_cordinacion_con_los_dirigentes_del_consejo_menor_deccasa_con_el_arqitecto_yeconomiista.jpg}%
    {Reuni\'{o}n de Coordinaci\'{o}n con Dirigentes del Consejo Menor de Ccasa}%
    {Oficina de Unidad Formuladora --- Municipalidad Distrital de Challhuahuacho}%
    {Febrero 2026}%
    {En las instalaciones de la Unidad Formuladora se sostuvo una reuni\'{o}n de coordinaci\'{o}n
     con los dirigentes del Consejo Menor de la Comunidad Campesina de Ccasa, contando con la
     presencia del arquitecto y el economista del equipo t\'{e}cnico. La reuni\'{o}n permiti\'{o}
     unificar criterios respecto al avance del proyecto, fortalecer el v\'{i}nculo institucional
     entre el equipo formulador y las autoridades comunales, y ratificar los compromisos asumidos
     por la comunidad como condici\'{o}n para la viabilidad del proyecto de inversi\'{o}n p\'{u}blica.}%

\end{document}