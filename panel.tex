\documentclass[12pt,a4paper]{article}

% Paquetes necesarios
\usepackage[utf8]{inputenc}
\usepackage[spanish]{babel}
\usepackage{graphicx}
\usepackage{geometry}
\usepackage{float}
\usepackage{caption}
\usepackage{subcaption}
\usepackage{fancyhdr}
\usepackage{xcolor}
\usepackage{tcolorbox}

% Configuración de márgenes
\geometry{
    a4paper,
    left=2.5cm,
    right=2.5cm,
    top=3cm,
    bottom=3cm
}

% Configuración de encabezado y pie de página
\pagestyle{fancy}
\fancyhf{}
\lhead{Panel Fotográfico}
\rhead{\today}
\cfoot{\thepage}

% Comando personalizado para insertar imagen con información
\newcommand{\panelfoto}[4]{
    \begin{tcolorbox}[colback=gray!5, colframe=gray!40, width=\textwidth, arc=3mm]
        \begin{figure}[H]
            \centering
            \includegraphics[width=0.8\textwidth, height=8cm, keepaspectratio]{#1}
        \end{figure}
        \vspace{-0.5cm}
        \begin{center}
            \textbf{Actividad:} #2 \\[0.3cm]
            \textbf{Fecha:} #3 \\[0.3cm]
            \textbf{Comentario:} #4
        \end{center}
    \end{tcolorbox}
    \vspace{0.5cm}
}

% Título del documento
\title{\Huge\textbf{Panel Fotográfico} \\ \Large Documentación de Actividades}
\author{JOSETRA44}
\date{\today}

\begin{document}

\maketitle
\thispagestyle{empty}
\newpage

\section*{Registro Fotográfico de Actividades}

% Ejemplo de uso de la plantilla
% Reemplaza los parámetros con tus propios valores:
% \panelfoto{ruta/imagen.jpg}{Nombre de la actividad}{DD/MM/AAAA}{Descripción o comentario}

\panelfoto{ejemplo1.jpg}{Actividad de ejemplo 1}{26/02/2026}{Este es un comentario de ejemplo para la primera imagen. Aquí se describe lo observado o realizado en la actividad.}

\panelfoto{ejemplo2.jpg}{Actividad de ejemplo 2}{25/02/2026}{Segundo comentario de ejemplo. Se pueden agregar detalles relevantes sobre el contexto de la fotografía.}

\panelfoto{ejemplo3.jpg}{Actividad de ejemplo 3}{24/02/2026}{Tercer comentario. Esta plantilla permite mantener un formato uniforme y profesional en todo el documento.}

% Agrega más fotografías según sea necesario usando el comando \panelfoto

\newpage
\section*{Notas de Uso}

Para agregar una nueva imagen al panel fotográfico, utiliza el siguiente comando:

\begin{verbatim}
\panelfoto{ruta/imagen.jpg}{Nombre Actividad}{Fecha}{Comentario}
\end{verbatim}

\textbf{Parámetros:}
\begin{itemize}
    \item \texttt{ruta/imagen.jpg}: Ruta del archivo de imagen
    \item \texttt{Nombre Actividad}: Descripción breve de la actividad
    \item \texttt{Fecha}: Fecha en formato DD/MM/AAAA
    \item \texttt{Comentario}: Observaciones y detalles relevantes
\end{itemize}

\end{document}